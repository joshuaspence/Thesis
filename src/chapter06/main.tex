There is still much work to be done in the area of my \thesis{} project. In
particular, from my experience with \gls{HLS} tools such as \AutoESL{}, it seems
that such tools are in somewhat of an early stage. One of the challenges that I
faced during the completion of this \thesis{} was a lack of available samples,
tutorials or otherwise for anything more than a basic, proof-of-concept type
\AutoESL{} project.

Nevertheless, I am optimistic with regards to the future of such tools, and with
regards to the \gls{HLS} process generally. I believe that the availability of
such tools, coupled with the increasing affordability of reconfigurable hardware
devices such as \glspl{FPGA} will provide many opportunities in the future for
reconfigurable hardware devices to evolve towards a \gls{SoC} device, as they
have already started to become. I believe that, one day, these new opportunities
may see reconfigurable hardware become more widespread and perhaps even used in
consumer devices, a suitable replacement for applications in which the
microprocessor does not adequately fit the purpose.

Throughout this project I gained an understanding of and familiarity with
hardware devices, particularly with \glspl{FPGA}. This was a valuable
opportunity for me to expand on my otherwise software-centric skill set.