\chapter{Introduction}
\label{ch:intro}

\section{Motivation}
\label{sec:motivation}
Anomaly detection is an important technique which can be applied to a wide 
range of applications. There are many techniques to detect and measure 
anomalies, each usually most applicable to some specific problem domain. A 
general algorithm for anomaly detection has proved difficult to design, due to 
various challenges associated with defining and measuring anomalies. Most 
existing approaches are based on statistical or geometrical measures involving 
distance. Whilst many algorithms work well for some subset of input data sets, 
it has been difficult to discover an algorithm which performs both correctly and
efficiently on all possible data sets.

Previous research has found merit in applying randomization techniques to highly
multivariate data sets in order to reduce the dimensionality of these data sets 
whilst maintaining their fundamental and statistical properties. Such reduction 
of the dimensionality of data, assuming it can be performed efficiently, allows 
previously-unscalable anomaly detection algorithms to be practically applied to 
a wider range of data and applications.

Anomaly detection is an important and contemporary problem in the field of 
computer science, and is of particular interesting to stock market analysis, 
network intrusion detection and image comparison.

\section{Contributions of this thesis}
\label{sec:contributions}
One key difficulty in anomaly detection is the efficient scaling of a general 
algorithm to apply to highly multivariate data. In this thesis, I explore the 
use of randomization techniques (such as random projections and commute time) 
in anomaly detection algorithms. These techniques provide encouraging results 
with regards to the run-time complexity of an algorithm.

Furthermore, in this thesis I attempt to make observations and analysis of the 
run-time of anomaly detection using randomization techniques, so as to identify
steps in the algorithms that bottleneck the algorithm's performance. Through 
this identification it would be possible to improve the \emph{actual} run-time 
performance of the algorithm by utilisation the advantages of reconfigurable 
computing.

\section{Organization}
\label{sec:organization}
The rest of this thesis is organized as follows. In \hyperref[ch:background]
{Chapter~\ref{ch:background}}, we provide background to various anomaly 
detection techniques, as well as randomization technqiues. In order to provide 
the reader with an understanding of the background topics, a brief background is
given to various topics in linear algebra, vector calculus and graph theory. In
\hyperref[ch:reconfigurableComputing]{Chapter~\ref{ch:reconfigurableComputing}}, 
we provide an overview of reconfigurable computing, including an explanation of 
Field-Programmable Gate Arrays (FPGAs). In \hyperref[ch:design]
{Chapter~\ref{ch:design}}, we profile the execution of an anomaly detection 
algorithm and explore possible improvements to the algorithm, in particular by 
outsourcing various stages of the algorithm to an FPGA device. In 
\hyperref[ch:implementation]{Chapter~\ref{ch:implementation}}, we describe the 
implementation of the improved algorithm and detail the process that was 
followed in order to construct the hardware processing device. In 
\hyperref[ch:results]{Chapter~\ref{ch:results}}, we record results obtained by 
benchmarking the device that was previously designed and constructed, and 
comparing the expected improvements to the algorithm's execution with the 
measured results. We conclude in \hyperref[ch:conclusions]
{Chapter~\ref{ch:conclusions}} by reflecting upon the results obtained through 
this research, and making suggestions for further research in this topic.
 
\section{Schedule}
\label{sec:schedule}

% Gantt chart
% See http://www.martin-kumm.de/tex_gantt_package.php
\begin{figure}
	\begin{center}
		\scalebox{0.5}{
		\begin{gantt}[
				xunitlength=0.5cm,
				fontsize=\small,
				titlefontsize=\small,
				drawledgerline=true
				]{18}{36}
			% Months
			\begin{ganttitle}
				\titleelement{Mar}{4}
				\titleelement{Apr}{4}
				\titleelement{May}{4}
				\titleelement{Jun}{4}
				\titleelement{Jul}{4}
				\titleelement{Aug}{4}
				\titleelement{Sep}{4}
				\titleelement{Oct}{4}
				\titleelement{Nov}{4}
			\end{ganttitle}
			% Split months into quarters
			\begin{ganttitle}
				\numtitle{1}{1}{4}{1}
				\numtitle{1}{1}{4}{1}
				\numtitle{1}{1}{4}{1}
				\numtitle{1}{1}{4}{1}
				\numtitle{1}{1}{4}{1}
				\numtitle{1}{1}{4}{1}
				\numtitle{1}{1}{4}{1}
				\numtitle{1}{1}{4}{1}
				\numtitle{1}{1}{4}{1}
			\end{ganttitle}

			% Tasks
			\ganttbar{Background Reading}{1}{6}
			\ganttbar{Literature Review}{1}{24}
			
			\ganttbar{Performance Benchmarking}{7}{1} \ganttcon{7}{2}{7}{4}
			\ganttbarcon{Algorithm Analysis}{8}{1}				
			\ganttbarcon{MATLAB-to-C code transformation}{9}{2}
			\ganttbarcon{Code Improvements}{11}{4}			
			
			\ganttbarcon{Design of FPGA hardware}{15}{8}		
			
			\ganttbar{Implementation of FPGA hardware}{22}{4}
			\ganttbar{Testing and Debugging}{25}{2}			
			\ganttbar{Performance Benchmarking}{26}{2}
			
			\ganttbar{Review}{23}{8}
			
			% Milestones
			\ganttmilestone{Topic Proposal Due}{6.68} % April 20
			\ganttmilestonecon{Progress Report Due}{12} % June 1
			\ganttmilestonecon{Draft Treatise Due}{29.56} % October 12
			\ganttmilestonecon{Final Treatise Due}{31.08} % October 24
			\ganttmilestonecon{Project Presentations}{33.2} % November 9
		\end{gantt}
		}
	\end{center}
\caption{Schedule for thesis work}
\end{figure}

