Anomaly detection is an important technique which can be applied to a wide range
of applications. There are many techniques to detect and measure anomalies, each
usually most applicable to some specific problem domain \citeNeeded{}. A general
algorithm for anomaly detection has proved difficult to design \citeNeeded{},
due to various challenges associated with defining and measuring anomalies. Most
existing approaches are based on statistical or geometrical measures involving
distance. Whilst many algorithms work well for some subset of input data sets,
it has been difficult to discover an algorithm which performs both correctly and
efficiently on all possible data sets.

Previous research has found merit in applying randomization techniques to highly
multivariate data sets in order to reduce the dimensionality of these data sets
whilst maintaining their fundamental and statistical properties. Such reduction
of the dimensionality of data, assuming it can be performed efficiently, allows
previously-unscalable anomaly detection algorithms to be practically applied to
a wider range of data and applications.

Anomaly detection is an important and contemporary problem in the field of
computer science, and is of particular interesting to stock market analysis,
network intrusion detection and image comparison.