\chapter{Reconfigurable Computing}
\label{ch:reconfigurableComputing}

\section{Introduction}
\label{sec:rcIntroduction}
In the computer and electronics world, computations are performed either in 
hardware or in software. Computer hardware provides highly optimized resources
for quickly performing critical tasks, but is permanently configured to a 
single task or application. Computer software offers a flexible approach, but is
orders of magnitude worse than a hardware implementation in terms of 
performance, silicon area efficiency and power consumption.

FPGAs are devices that combine the advantages of hardware implementations with 
the flexibility of software implementations. The computations are programmed 
into the silicon chip such that an FPGA system can be programmed and 
reprogrammed many times. The utility of FPGAs does, however, come at a price. 
Whilst, compared to a microprocessor, FPGAs are typically several orders of 
magnitude faster and more power efficient, the task of creating efficient 
programs for these devices is difficult.

Typically, FPGAs are useful only for operations that process large streams of 
data, such as signal processing, networking, and the like. Compared to 
integrated circuit, they may be 5 to 25 times worse in terms of area, delay, and
performance \cite{HAUCK08}. However, while an integrated circuit design may take
months to years to develop and have a multimillion-dollar price tag, an FPGA 
design might only take days to create and cost tens to hundreds of dollars.

\section{Field-Programmable Gate Array (FPGA)}
\label{sec:FPGA}
