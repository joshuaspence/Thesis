%%%%%%%%%%%%%%%%%%%%%%%%%%%%%%%%%%%%%%%%%%%%%%%%%%%%%%%%%%%%%%%%%%%%%%%%%%%%%%%%
% VHSIC Hardware Description Language
%%%%%%%%%%%%%%%%%%%%%%%%%%%%%%%%%%%%%%%%%%%%%%%%%%%%%%%%%%%%%%%%%%%%%%%%%%%%%%%%
\subsection{VHSIC Hardware Description Language}
\label{vhdl}

% Introduction
\subsubsection{Introduction}
\label{vhdl:introduction}
% TODO
Modern \gls{FPGA} devices can many circuit elements, often comprising hundreds
of \glspl{LUT}, hundreds of embedded memories and hundreds of multipliers.
Programming these elements individually would be an intractable task for the
designer, and so instead it is common for the designer to describe the design in
terms of logical expressions, and then allowing for synthesis and layout tools
to program the individual \gls{FPGA} elements \cite{Hauck:2003}.

\gls{RTL} is a popular discipline for describing an \gls{FPGA}'s logical
expressions, allowing the design to describe the logic between each pair of
register stages \cite{Hauck:2003}. \gls{VHDL} is a popular programming language
that supports \gls{RTL} hardware descriptions. Verilog (IEEE 1364) is another
popular hardware descriptive programming language that offers similar
functionality to \gls{VHDL}. \gls{VHDL} and Verilog differ mainly in syntax ---
\gls{VHDL} is derived from Ada, whereas Verilog is derived from C. However, both
languages are \gls{IEEE} standards and are periodically reviewed to ensure that
they comply to industry expectations.

% Structural Description
\subsubsection{Structural Description}
\label{vhdl:structuralDescription}
% TODO

% RTL Description
\subsubsection{RTL Description}
\label{vhdl:rtlDescription}
% TODO

% Parametric Hardware Generation
\subsubsection{Parametric Hardware Generation}
\label{vhdl:parametricHardwareGeneration}
% TODO

% Limitations
\subsubsection{Limitations}
\label{vhdl:limitations}
% TODO

%%%%%%%%%%%%%%%%%%%%%%%%%%%%%%%%%%%%%%%%%%%%%%%%%%%%%%%%%%%%%%%%%%%%%%%%%%%%%%%%
% Verilog (IEEE 1364)
%%%%%%%%%%%%%%%%%%%%%%%%%%%%%%%%%%%%%%%%%%%%%%%%%%%%%%%%%%%%%%%%%%%%%%%%%%%%%%%%
\subsection{Verilog (IEEE 1364)}
\label{verilog}
% TODO

%%%%%%%%%%%%%%%%%%%%%%%%%%%%%%%%%%%%%%%%%%%%%%%%%%%%%%%%%%%%%%%%%%%%%%%%%%%%%%%%
% Hardware Compilation Flow
%%%%%%%%%%%%%%%%%%%%%%%%%%%%%%%%%%%%%%%%%%%%%%%%%%%%%%%%%%%%%%%%%%%%%%%%%%%%%%%%
\subsection{Hardware Compilation Flow}
\label{hardwareCompilationFlow}
% TODO

% Constraints
\subsubsection{Constraints}
\label{hardwareCompilationFlow:constraints}
% TODO