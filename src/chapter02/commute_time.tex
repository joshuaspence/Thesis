% INTRODUCTION
\subsection{Introduction}
\label{sec:commuteTimeIntroduction}
Commute time is a robust distance metric derived from a random walk on graphs 
\cite{Khoa:2012}. In \citetitle{Khoa:2012}, \citeauthor{Khoa:2012} demonstrated 
how commute time can be used as a distance measure for data mining tasks such as
anomaly detection and clustering. A prohibitive limitation of this technique is 
that the calculation of commute time involves the eigen decomposition of the 
graph Laplacian, making it impractical for large graphs.

A major advantage of using commute time as a distance metric for outlier 
detection is that it effectively captures not only the distances between data 
points but also the density of the data \citeNeeded{}. This property results in 
a distance metric that can be effectively used to capture global, local and 
group anomalies.

The commute time between two nodes $i$ and $j$ in a graph is the number of steps
that a random walk, starting from $i$ will take to visit $j$ and then come back 
to $i$ for the first time. The fact that the commute time is averaged over all 
paths (and not just the shortest path) makes it more robust to data 
perturbations and it can also capture graph density \cite{Khoa:2012}. Since it 
is a measure which can capture the geometrical structure of the data and is 
robust to noise, commute time can be applied in methods where Euclidean or other
distances are used and thus the limitations of these metrics can be avoided.

% LIMITATIONS
\subsection{Limitations}
\label{sec:commuteTime:limitations}
The computation of commute time requires the eigen decomposition (see 
\autoref{sec:eigenDecomposition}) of the graph Laplacian matrix (see 
\autoref{sec:laplacianMatrices}), a computation which takes $O(n^{3})$ time and 
thus is not practical for large graphs \citeNeeded{}. Methods to approximate the
commute time to reduce the computational time are required in order to 
efficiently use commute time in large datasets.

% ANOMALY DETECTION USING COMMUTE TIME
\subsection{Anomaly Detection Using Commute Time}
\label{sec:anomalyDetectionUsingCommuteTime}
% TODO
