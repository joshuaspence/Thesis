Using \software{MATLAB}'s \command{profile} command, I was able to analyse and
profile the \algm{anomaly detection using commute distance} algorithm and make
an assessment of the running time of the algorithm. By analysing the code in
order to determine which regions are most limiting to the algorithm's
performance, I was able to determine an ideal candidate function for a hardware
implementation. The results of the algorithm profiling appear in the following
sections.

%%%%%%%%%%%%%%%%%%%%%%%%%%%%%%%%%%%%%%%%%%%%%%%%%%%%%%%%%%%%%%%%%%%%%%%%%%%%%%%%
% Method
%%%%%%%%%%%%%%%%%%%%%%%%%%%%%%%%%%%%%%%%%%%%%%%%%%%%%%%%%%%%%%%%%%%%%%%%%%%%%%%%
\section{Method}
\label{software:algorithmProfiling:matlab:method}
In order to profile the provided \software{MATLAB} code, it was necessary to
gain a basic understanding of the structure of the source code, as well as
insight into how the algorithm is intended to be used (i.e.\ pre-conditions,
post-conditions, etc.).

In order to facilitate later testing of other algorithm implementations, the
decision was made to modify the original code so as to provide more verbose
logging, as well as ensuring that all variables and plots that were produced by
the algorithm were stored on the file system. Furthermore, the original
\command{TopN_Outlier_Pruning_Block} function was revised, adding minor
optimizations as well as improving the code's aesthetics. The \software{MATLAB}
code for the \command{TopN_Outlier_Pruning_Block} function that was used for
profiling can be found in \autoref{sourceCode:matlab}.

The algorithm profiling was performed on all data set instances so as to gain a
comprehensive insight into the strengths and deficiencies of the chosen
algorithm. The \software{MATLAB} \command{profile} command records information
regarding execution time, number of calls, parent functions, child functions,
code line hit count and code line execution time. This data is collected in
real-time during the algorithm's execution and is prepared into a report at
program completion. The timing information that is recorded by the
\command{profile} command relates to \gls{CPU} time --- that is, to the amount
of time for which the \gls{CPU} was used for processing instructions of the
code. This timing information, contrary to \emph{real} time, discounts idle
\gls{CPU} cycles that are caused as a result of \gls{IO} operations.

%%%%%%%%%%%%%%%%%%%%%%%%%%%%%%%%%%%%%%%%%%%%%%%%%%%%%%%%%%%%%%%%%%%%%%%%%%%%%%%%
% Results
%%%%%%%%%%%%%%%%%%%%%%%%%%%%%%%%%%%%%%%%%%%%%%%%%%%%%%%%%%%%%%%%%%%%%%%%%%%%%%%%
\section{Results}
\label{software:algorithmProfiling:matlab:results}
The following plots compare the `self time' of functions comprising the
\algm{anomaly detection using commute distance} algorithm. Self time is a more
relevant metric than total time, as total time is a cumulative measure that
includes all self times of functions called by some outer function. Self time,
on the other hand, is the time spent in a function excluding the time spent in
its child functions. Self time also includes overhead resulting from the process
of profiling.

It is my aim to select a single function as a candidate for a hardware
implementation, selecting the function that has the most significant impact on
the total algorithm execution time. As such, functions with a relatively small
self time need not be considered, and have excluded from the following plots.
Specifically, any function with a self time less than 3\% of the total execution
time has been aggregated and is represented in the plots as ``Other'' functions.

The legend shown in \autoref{profiling:matlab:legend} is common amongst all of
the \software{MATLAB} profiling results plots shown in
\autoref{fig:profiling:matlab}.

\begin{figure}
    \centering
    \pielegend[bound,radius=1.5,color={blue!60,violet!60,orange!60,lime!60,blue!60,black!60,cyan!60!yellow!60,magenta!60,magenta!60,white!60,blue!60!cyan!60,red!60,magenta!60,pink!60,blue!60,blue!60,blue!60,green!60,orange!60!cyan!60,red!60!cyan!60,yellow!60,cyan!60,red!60!cyan!60,green!60,purple!60,red!60!blue!60}]{
1/\escape{TopN_Outlier_Pruning_Block},
1/\escape{findobjhelper},
1/\escape{isprop},
1/\escape{sim},
1/\escape{TopN_Outlier_Pruning_Block},
1/\escape{ObserveAxesNextPlot},
1/\escape{knn_components_sparse},
1/\escape{fireprintbehavior},
1/\escape{mx_d_preconditioner},
1/\escape{Other},
1/\escape{drawGraph},
1/\escape{localPeek},
1/\escape{newplot},
1/\escape{restorehg},
1/\escape{TopN_Outlier_Pruning_Block},
1/\escape{TopN_Outlier_Pruning_Block},
1/\escape{TopN_Outlier_Pruning_Block},
1/\escape{lineseries},
1/\escape{@(b)mx_d_preconditioner(sH,b)},
1/\escape{hardcopy},
1/\escape{hggetbehavior},
1/\escape{pcg},
1/\escape{prepare},
1/\escape{iterapp},
1/\escape{princomp},
1/\escape{kdtree_k_nearest_neighbors}
}

    \caption{Common legend for MATLAB profiling plots}
    \label{profiling:matlab:legend}
\end{figure}
\begin{pieplots}{fig:profiling:matlab}{MATLAB profiling plots}
\pieplot{\escape{testCDST}}{\pie[bound,text=none,radius=1.5,color={blue!60,white!60}]{
79.76/\escape{TopN_Outlier_Pruning_Block},
20.24/\escape{Other}
}
}
\pieplot{\escape{spam}}{\pie[bound,text=none,radius=1.5,color={cyan!60,blue!60,cyan!60!yellow!60,white!60}]{
3.96/\escape{pcg},
73.03/\escape{TopN_Outlier_Pruning_Block},
4.56/\escape{knn_components_sparse},
18.46/\escape{Other}
}
}
\pieplot{\escape{testCD}}{\pie[bound,text=none,radius=1.5,color={magenta!60,orange!60,blue!60,white!60}]{
3.09/\escape{newplot},
4.32/\escape{isprop},
62.28/\escape{TopN_Outlier_Pruning_Block},
30.31/\escape{Other}
}
}
\pieplot{\escape{runningex30k}}{\pie[bound,text=none,radius=1.5,color={blue!60,white!60}]{
97.01/\escape{TopN_Outlier_Pruning_Block},
2.99/\escape{Other}
}
}
\pieplot{\escape{testCDST2}}{\pie[bound,text=none,radius=1.5,color={blue!60,white!60}]{
81.67/\escape{TopN_Outlier_Pruning_Block},
18.33/\escape{Other}
}
}
\pieplot{\escape{segmentation}}{\pie[bound,text=none,radius=1.5,color={blue!60,white!60,red!60!cyan!60}]{
70.72/\escape{TopN_Outlier_Pruning_Block},
25.98/\escape{Other},
3.30/\escape{prepare}
}
}
\pieplot{\escape{magicgamma}}{\pie[bound,text=none,radius=1.5,color={blue!60,white!60,cyan!60!yellow!60}]{
85.12/\escape{TopN_Outlier_Pruning_Block},
6.41/\escape{Other},
8.47/\escape{knn_components_sparse}
}
}
\pieplot{\escape{ball1}}{\pie[bound,text=none,radius=1.5,color={cyan!60,yellow!60,blue!60,white!60,orange!60,red!60,blue!60!cyan!60}]{
3.61/\escape{pcg},
3.23/\escape{hggetbehavior},
56.96/\escape{TopN_Outlier_Pruning_Block},
24.19/\escape{Other},
5.38/\escape{isprop},
3.34/\escape{localPeek},
3.30/\escape{drawGraph}
}
}
\pieplot{\escape{pendigits}}{\pie[bound,text=none,radius=1.5,color={blue!60,cyan!60!yellow!60,white!60}]{
84.51/\escape{TopN_Outlier_Pruning_Block},
6.01/\escape{knn_components_sparse},
9.48/\escape{Other}
}
}
\pieplot{\escape{runningex20k}}{\pie[bound,text=none,radius=1.5,color={blue!60,white!60}]{
96.44/\escape{TopN_Outlier_Pruning_Block},
3.56/\escape{Other}
}
}
\pieplot{\escape{musk}}{\pie[bound,text=none,radius=1.5,color={blue!60,cyan!60!yellow!60,white!60,red!60!blue!60}]{
82.39/\escape{TopN_Outlier_Pruning_Block},
3.13/\escape{knn_components_sparse},
9.48/\escape{Other},
5.00/\escape{kdtree_k_nearest_neighbors}
}
}
\pieplot{\escape{runningex10k}}{\pie[bound,text=none,radius=1.5,color={blue!60,white!60}]{
94.57/\escape{TopN_Outlier_Pruning_Block},
5.43/\escape{Other}
}
}
\pieplot{\escape{runningex40k}}{\pie[bound,text=none,radius=1.5,color={blue!60,white!60}]{
96.98/\escape{TopN_Outlier_Pruning_Block},
3.02/\escape{Other}
}
}
\pieplot{\escape{spam_train}}{\pie[bound,text=none,radius=1.5,color={cyan!60,blue!60,white!60,cyan!60!yellow!60}]{
4.71/\escape{pcg},
74.51/\escape{TopN_Outlier_Pruning_Block},
16.64/\escape{Other},
4.14/\escape{knn_components_sparse}
}
}
\pieplot{\escape{testoutrank}}{\pie[bound,text=none,radius=1.5,color={cyan!60,orange!60!cyan!60,orange!60,blue!60,green!60,magenta!60,white!60}]{
19.31/\escape{pcg},
3.31/\escape{@(b)mx_d_preconditioner(sH,b)},
4.38/\escape{isprop},
31.94/\escape{TopN_Outlier_Pruning_Block},
10.89/\escape{iterapp},
4.15/\escape{mx_d_preconditioner},
26.02/\escape{Other}
}
}
\pieplot{\escape{testCDST3}}{\pie[bound,text=none,radius=1.5,color={blue!60,white!60}]{
80.17/\escape{TopN_Outlier_Pruning_Block},
19.83/\escape{Other}
}
}
\pieplot{\escape{runningex1k}}{\pie[bound,text=none,radius=1.5,color={orange!60,blue!60,white!60}]{
3.57/\escape{isprop},
71.80/\escape{TopN_Outlier_Pruning_Block},
24.62/\escape{Other}
}
}
\pieplot{\escape{mesh_network}}{\pie[bound,text=none,radius=1.5,color={blue!60,white!60,red!60!blue!60,purple!60}]{
68.01/\escape{TopN_Outlier_Pruning_Block},
23.76/\escape{Other},
5.15/\escape{kdtree_k_nearest_neighbors},
3.09/\escape{princomp}
}
}
\pieplot{\escape{letter-recognition}}{\pie[bound,text=none,radius=1.5,color={blue!60,white!60,cyan!60!yellow!60}]{
84.26/\escape{TopN_Outlier_Pruning_Block},
8.74/\escape{Other},
7.00/\escape{knn_components_sparse}
}
}
\pieplot{\escape{connect4}}{\pie[bound,text=none,radius=1.5,color={blue!60,white!60,red!60!blue!60,cyan!60!yellow!60}]{
85.22/\escape{TopN_Outlier_Pruning_Block},
3.28/\escape{Other},
3.73/\escape{kdtree_k_nearest_neighbors},
7.77/\escape{knn_components_sparse}
}
}
\pieplot{\escape{runningex50k}}{\pie[bound,text=none,radius=1.5,color={blue!60,white!60}]{
96.88/\escape{TopN_Outlier_Pruning_Block},
3.12/\escape{Other}
}
}
\end{pieplots}


%%%%%%%%%%%%%%%%%%%%%%%%%%%%%%%%%%%%%%%%%%%%%%%%%%%%%%%%%%%%%%%%%%%%%%%%%%%%%%%%
% Performance Bottleneck
%%%%%%%%%%%%%%%%%%%%%%%%%%%%%%%%%%%%%%%%%%%%%%%%%%%%%%%%%%%%%%%%%%%%%%%%%%%%%%%%
\subsection{Performance Bottleneck}
\label{algorithmPerformance:bottleneck}
From observations of the results of the algorithm profiling, it was observed
that the performance of the `anomaly detection using commute-distance' algorithm
is bottle-necked significantly by a function named
\command{TopN_Outlier_Pruning_Block}. The MATLAB code for this function can be
found in \autoref{sourceCode:matlab}. Analysis of this function, as well as
discussions with \citeauthor{Khoa:2012} revealed that the algorithm was
originally devised by \citeauthor{Bay:2003} and published in the paper
\citetitle{Bay:2003}. The general steps of the algorithm are described in
\autoref{algm:TopNOutlierPruningBlock}.

\begin{algorithm}
    \LinesNumbered

\SetKwInput{InputK}{k}\InputK{the number of nearest neighbours to consider}
\SetKwInput{InputN}{N}\InputN{the number of outliers to return}
\SetKwInput{InputData}{data}\InputData{a set of examples in random order}
\SetKwInOut{Output}{outliers}\Output{a set of outliers}
\SetKwInOut{OutputScores}{outlier\_scores}\OutputScores{}

\SetKwData{BlockIterator}{b}
\SetKwData{Block}{block}
\SetKwData{Cutoff}{cutoff}
\SetKwData{DataIterator}{d}
\SetKwData{Data}{data}
\SetKwData{NumK}{k}
\SetKwData{NumN}{N}
\SetKwData{Neighbours}{neighbours}
\SetKwData{OutlierIterator}{o}
\SetKwData{Outliers}{outliers}
\SetKwData{Score}{score}

\SetKwFunction{Closest}{closest}
\SetKwFunction{Distance}{distance}
\SetKwFunction{LoadBlock}{load\_next\_block}
\SetKwFunction{MaxDist}{max\_dist}
\SetKwFunction{Min}{min}
\SetKwFunction{Top}{top}

\Begin{
    \Cutoff $\leftarrow$ 0\tcp*[l]{set the cutoff for pruning to $0$}
    \Outliers $\leftarrow\emptyset$\tcp*[l]{initialize to the empty set}
    \BlankLine
    \While(\tcp*[h]{load a block of examples from \Data}){\Block $\leftarrow$ \LoadBlock{\Data}}{
        \ForEach{\BlockIterator $\in$ \Block}{
            $\Neighbours[\BlockIterator]\leftarrow\emptyset$\;
        }
        \BlankLine
        \ForEach{\DataIterator $\in$ \Data}{
            \ForEach{\BlockIterator $\in$ \Block $:$ \BlockIterator $\neq$ \DataIterator}{
                \If{$|\Neighbours[\BlockIterator]|<$ \NumK $\:\lor\:$ \Distance{\BlockIterator, \DataIterator} $<$ \MaxDist{\BlockIterator, $\Neighbours[\BlockIterator]$}}{
                    %$\Neighbours[\BlockIterator] \leftarrow$ \Closest{\BlockIterator, $\Neighbours[\BlockIterator]$} $\cup$ \DataIterator,\NumK)$\;
                    \If{$\Score[\Neighbours[\BlockIterator],\BlockIterator)<\Cutoff$}{
                        \Block $\leftarrow$ \Block $\setminus$ \BlockIterator\;
                    }
                }
            }
        }
        \BlankLine
        \Outliers $\leftarrow$ \Top{\Block $\cup$ \Outliers, \NumN}\;
        \Cutoff $\leftarrow$ $\Min_{\OutlierIterator \in \Outliers}(\Score(\OutlierIterator))$\tcp*[l]{the cutoff is the score of the weakest outlier}
    }

    \BlankLine
    \KwRet{\Outliers}\;
}

    \caption{TopN\_Outlier\_Pruning\_Block}
    \label{algm:TopNOutlierPruningBlock}
\end{algorithm}

In this algorithm, the \command{score} function can be any monotonically
decreasing function of the nearest neighbour distances such as the distance to
the $k$th nearest neighbour, or the average distance to the $k$ neighbours
\cite{Bay:2003}.

The main idea of the nested loop algorithm is that for each example in the
input set \verb+Data+, the algorithm keeps track of the closest neighbours found
so far. When an example's closest neighbours achieve a score lower than the
\verb+cutoff+, the example is discarded because it can no longer be an outlier.
As more examples are processed, the algorithm finds more extreme outliers and
the \verb+cutoff+ increases along with pruning efficiency \cite{Bay:2003}.

In the worst case, the performance of the algorithm is very poor --- due to the
nested loops, it could require $O(N^{2})$ distance computations and
$O(\frac{N}{blocksize} \times N)$ data accesses. However, \citeauthor{Bay:2003}
proved (through application of the algorithm to both real and synthetic data
sets) that the algorithm performs considerably better than the expected
$O(N^{2})$ running time in the average case. The performance improvements over
similar algorithms were attributed to the application of randomization and
pruning techniques. The outlier pruning problem can be considered similar to the
problem of conducting a set of independent Bernoulli trials in which examples
are analysed until $k$ examples within distance $d$ are found, or until the data
set is exhausted. \citeauthor{Bay:2003} proved that the number of trials
expected to achieve $k$ examples within distance $d$ is given by:

\begin{equation}
    E[Y] \leq \frac{k}{\pi(\textbf{x})} + \Bigg(1-\sum_{y=k}^{N} P(Y=y)\Bigg) \times N
    \label{eqn:outlierPruneComplexity}
\end{equation}

Where $\pi(x)$ is the probability that a random drawn example lies within
distance $d$ of point $x$ and $P(Y=y)$ is the probability of obtaining the $k$th
success on trial $y$.

The first term in \autoref{eqn:outlierPruneComplexity} represents the number of
distance computations to eliminate non-outliers, and does not depend on $N$. The
second term represents the expected cost of outliers, and does depend on $N$,
yielding an overall quadratic dependency to process $N$ examples in total.
However, note that we typically set the program parameters to return a small an
possibly fixed number of outliers. Thus the first term dominates and we obtain
near linear performance \cite{Bay:2003}. More specifically, it was determined
that the primary factor determining the scaling is how the cut-off changes as
$N$ increases.

There are, however, some limitations of the aforementioned algorithm.
Specifically:
\begin{enumerate}
    \item The algorithm assumes that the data is in random order. If the data is
        not in random order and is sorted then the performance can be poor.
    \item The algorithm depends on the independence of examples. If examples are
        dependent in such a way that they have similar values (and will likely
        be in the set of $k$ nearest neighbours) this can cause performance to
        be poor as the algorithm may have to scan the entire data set to find
        the dependent examples.
    \item The algorithm can perform poorly occurs when the data does not contain
        outliers.
\end{enumerate}

% TODO