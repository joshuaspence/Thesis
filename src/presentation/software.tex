%%%%%%%%%%%%%%%%%%%%%%%%%%%%%%%%%%%%%%%%%%%%%%%%%%%%%%%%%%%%%%%%%%%%%%%%%%%%%%%%
% Method
%%%%%%%%%%%%%%%%%%%%%%%%%%%%%%%%%%%%%%%%%%%%%%%%%%%%%%%%%%%%%%%%%%%%%%%%%%%%%%%%
\subsection{Method}
\begin{frame}<1-2>[label=method]{Method}
    \note<1>{And so, just to provide a quick outline of the various
        implementation stages that were iterated through in order to realise my
        hardware design.}

    \begin{enumerate}
        \item<2-> MATLAB source code
        \note<2>{The initial implementation of the algorithm, as provided by
            Khoa, comprised of MATLAB source code. This code formed the basis of
            my analysis and benchmarking.}

        \bigskip
        \item<3-> MATLAB source code + MEX file
        \note<3>{Following from this, a subset of the MATLAB source code (namely
            that of the function that seemed most promising for a hardware
            implementation) was rewritten as C++ source code using MEX header
            files to interact with MATLAB.}

        \bigskip
        \item<4-> C++ source code
        \note<4>{The next stage involved the complete separation of the chosen
            function from the algorithm framework in MATLAB. The source code was
            rewritten as pure C++ and additional wrappers were created to
            facilitate isolated testing.}

        \bigskip
        \item<5-> AutoESL high level synthesis
        \note<5>{Further analysis of the C++ function implementation provided
            additional insight into the algorithm's key properties and
            weaknesses, from which a smaller subset of source code was selected
            for high level synthesis --- a process to convert C++ source code
            into a low level hardware description language.}

        \bigskip
        \item<6-> Hardware design
        \note<6>{Following from the high level synthesis, and using the Xilinx
            design suite, I was able to transform my high-level software design
            into a low-level, parallelised hardware design.}
    \end{enumerate}
\end{frame}

\begin{frame}{Algorithm Profiling (MATLAB)}{\escape{commute_distance_anomaly}}
    \profilingPlots{matlab}{testoutrank}{runningex40k}{connect4}

    \note{These graphs show the results of initial algorithm profiling from the
        MATLAB source code. For every tested data set, the largest function in
        terms of execution timed was named \escape{TopN_Outlier_Pruning_Block},
        and involved a sophisticated nested-for loops method of discovering
        outliers in a data set of fixed dimensionality.}
\end{frame}

\againframe<3-4>{method}

\begin{frame}{Algorithm Profiling (C)}{\escape{TopN_Outlier_Pruning_Block}}
    \profilingPlots{c}{testoutrank}{runningex40k}{connect4}

    \note{These graphs show the relative execution times of functions comprising
        the \escape{TopN_Outlier_pruning_Block} function, executed on 3
        different data sets.

        These graphs shows that the vast majority of the algorithm's execution
        time is attributed to the \escape{distance_squared} function --- a
        simple function which returns the square of the Euclidean distance
        between two given vectors.}
\end{frame}

\againframe<5->{method}
