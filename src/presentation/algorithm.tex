%%%%%%%%%%%%%%%%%%%%%%%%%%%%%%%%%%%%%%%%%%%%%%%%%%%%%%%%%%%%%%%%%%%%%%%%%%%%%%%%
% Introduction
%%%%%%%%%%%%%%%%%%%%%%%%%%%%%%%%%%%%%%%%%%%%%%%%%%%%%%%%%%%%%%%%%%%%%%%%%%%%%%%%
\subsection{Introduction}
\begin{frame}<1-3>[label=algorithm-introduction]{Introduction}
        {Anomaly Detection Using Commute Time}
    {\Large \textbf{Anomaly Detection Using Commute Time}}\\

    \medskip
    {\small \fullcite{Khoa:2012}}
    \note<1>{The algorithm that was selected for implementation was developed by
        \citeauthor{Khoa:2012} and published in \citetitle{Khoa:2012}.}

    \medskip
    \begin{columns}[c]
        \column{0.5\textwidth}
        \begin{figure}[H]
            \includegraphics[width=0.7\textwidth]{algorithm/run-time}
            \caption{\scriptsize Execution time of approximate commute time method versus
                Euclidean distance and \gls{LOF} methods}
        \end{figure}

        \column{0.5\textwidth}
        \begin{figure}[H]
            \includegraphics[width=0.7\textwidth]{algorithm/accuracy}
            \caption{\scriptsize Accuracy and execution time of approximated commute
                distance method compared to non-approximated method}
        \end{figure}
    \end{columns}

    \note<1>{These graphs show the measured performance of the chosen algorithm.}
    \note<2>{The left graph shows how run time (using \emph{approximated}
        commute time) is proportional to $O(n \log n)$, compared to:
        \begin{itemize}
            \item $O(n)$ using Euclidean distance
            \item $O(n^2)$ using \gls{LOF}
        \end{itemize}}
    \note<3>{The right graph shows how anomaly detection using
        \emph{approximated} commute time preserves a high percentage (84.6\% on
        average) of the top anomalies discovered without using approximations.}
\end{frame}

%%%%%%%%%%%%%%%%%%%%%%%%%%%%%%%%%%%%%%%%%%%%%%%%%%%%%%%%%%%%%%%%%%%%%%%%%%%%%%%%
% The Algorithm
%%%%%%%%%%%%%%%%%%%%%%%%%%%%%%%%%%%%%%%%%%%%%%%%%%%%%%%%%%%%%%%%%%%%%%%%%%%%%%%%
\subsection{The Algorithm}
\begin{frame}[label=algorithm]{The Algorithm}
        {Anomaly Detection Using Commute Time}\relax
    {\tiny
        \begin{algorithm}[H]
            % TODO
\LinesNumbered

\SetKwInput{InputK}{k}
\SetKwInput{InputN}{N}
\SetKwInput{InputD}{Data}
\SetKwInOut{OutputO}{outliers}

\InputK{the number of nearest neighbors}
\InputN{the number of outliers to return}
\InputD{a set of examples in random order}
\OutputO{a set of outliers}

\SetKwData{varB}{b}
\SetKwData{Block}{block}
\SetKwData{Cutoff}{cutoff}
\SetKwData{varD}{d}
\SetKwData{Data}{Data}
\SetKwData{varK}{k}
\SetKwData{varN}{N}
\SetKwData{Neighbours}{neighbours}
\SetKwData{varO}{o}
\SetKwData{Outliers}{outliers}
\SetKwData{Score}{score}

\SetKwFunction{Closest}{closest}
\SetKwFunction{Distance}{distance}
\SetKwFunction{GetNextBlock}{getNextBlock}
\SetKwFunction{MaxDist}{maxDist}
\SetKwFunction{Min}{min}
\SetKwFunction{Top}{top}

\Begin{
    $\Cutoff \longleftarrow 0$\tcp*[l]{set the cutoff for pruning to $0$}
    $\Outliers \longleftarrow \emptyset$\tcp*[l]{initialize to the empty set}
    \BlankLine
    \While(\tcp*[h]{load a block of examples from \varD}){$\Block \longleftarrow \GetNextBlock{\Data}$}{
        $\Neighbours(\varB) \longleftarrow \emptyset, \quad \forall \, \varB \in \Block$\;
        \BlankLine
        \ForEach{$\varD \in \Data$}{
            \ForEach{$\varB \in \Block \: : \: \varB \neq \varD$}{
                \If{$|\Neighbours(\varB)| < \varK \: \lor \: \Distance{\varB, \varD} < \MaxDist{\varB, \Neighbours(\varB)}$}{
                    $\Neighbours(\varB) \longleftarrow \Closest{\varB, \Neighbours(\varB)} \cup \varD, \varK)$\;
                    \If{$\Score(\Neighbours(\varB), \varB) < \Cutoff$}{
                        $\Block \longleftarrow \Block \setminus \varB$\;
                    }
                }
            }
        }
        \BlankLine
        $\Outliers \longleftarrow $\Top{$\Block \cup \Outliers$, $\varN$}\tcp*[l]{keep only the top n outliers}
        $\Cutoff \longleftarrow \Min_{\varO \in \Outliers}(\Score(\varO))$\tcp*[l]{the cutoff is the score of the weakest outlier}
    }
    \KwRet{\Outliers}\;
}

        \end{algorithm}
    }

    \note<1>{The anomaly detection algorithm is described in the next two
        slides, however due to time constraints I will not discuss the algorithm
        in any great detail. Basically, the algorithm consists of:
    }

    \note<2>{Constructing the $k$-nearest neighbour graph for the input data
        set.}
    \note<3>{Using graph component sampling and eigenspace embedding to reduce
        the dimensionality of the data.}
    \note<4>{Reconstructing the $k$-nearest neighbour graph using the sampled
        data.}
    \note<5>{Computing the Laplacian matrix of the sampled graph and its $m$
        smallest eigenvectors.}
    \note<6>{Performing distance based anomaly detection.}

    \note<6>{From analysis of the run time performance of this anomaly detection
        algorithm, to be discussed later, the
        ``distance based anomaly detection'' stage of the algorithm was selected
        as a canditate for a hardware implementation.}
\end{frame}

\begin{frame}[label=algorithm]{The Algorithm}
        {Outlier Pruning}\relax
    {\tiny
        \begin{algorithm}[H]
            \LinesNumbered

\SetKwInput{InputK}{k}\InputK{the number of nearest neighbours to consider}
\SetKwInput{InputN}{N}\InputN{the number of outliers to return}
\SetKwInput{InputData}{data}\InputData{a set of examples in random order}
\SetKwInOut{Output}{outliers}\Output{a set of outliers}
\SetKwInOut{OutputScores}{outlier\_scores}\OutputScores{}

\SetKwData{BlockIterator}{b}
\SetKwData{Block}{block}
\SetKwData{Cutoff}{cutoff}
\SetKwData{DataIterator}{d}
\SetKwData{Data}{data}
\SetKwData{NumK}{k}
\SetKwData{NumN}{N}
\SetKwData{Neighbours}{neighbours}
\SetKwData{OutlierIterator}{o}
\SetKwData{Outliers}{outliers}
\SetKwData{Score}{score}

\SetKwFunction{Closest}{closest}
\SetKwFunction{Distance}{distance}
\SetKwFunction{LoadBlock}{load\_next\_block}
\SetKwFunction{MaxDist}{max\_dist}
\SetKwFunction{Min}{min}
\SetKwFunction{Top}{top}

\Begin{
    \Cutoff $\leftarrow$ 0\tcp*[l]{set the cutoff for pruning to $0$}
    \Outliers $\leftarrow\emptyset$\tcp*[l]{initialize to the empty set}
    \BlankLine
    \While(\tcp*[h]{load a block of examples from \Data}){\Block $\leftarrow$ \LoadBlock{\Data}}{
        \ForEach{\BlockIterator $\in$ \Block}{
            $\Neighbours[\BlockIterator]\leftarrow\emptyset$\;
        }
        \BlankLine
        \ForEach{\DataIterator $\in$ \Data}{
            \ForEach{\BlockIterator $\in$ \Block $:$ \BlockIterator $\neq$ \DataIterator}{
                \If{$|\Neighbours[\BlockIterator]|<$ \NumK $\:\lor\:$ \Distance{\BlockIterator, \DataIterator} $<$ \MaxDist{\BlockIterator, $\Neighbours[\BlockIterator]$}}{
                    %$\Neighbours[\BlockIterator] \leftarrow$ \Closest{\BlockIterator, $\Neighbours[\BlockIterator]$} $\cup$ \DataIterator,\NumK)$\;
                    \If{$\Score[\Neighbours[\BlockIterator],\BlockIterator)<\Cutoff$}{
                        \Block $\leftarrow$ \Block $\setminus$ \BlockIterator\;
                    }
                }
            }
        }
        \BlankLine
        \Outliers $\leftarrow$ \Top{\Block $\cup$ \Outliers, \NumN}\;
        \Cutoff $\leftarrow$ $\Min_{\OutlierIterator \in \Outliers}(\Score(\OutlierIterator))$\tcp*[l]{the cutoff is the score of the weakest outlier}
    }

    \BlankLine
    \KwRet{\Outliers}\;
}

        \end{algorithm}
    }

    \note{This slide describes the distance based anomaly detection method that
        is to be implemented in hardware. To aide in the understanding of this
        function, I will demonstrate an animation which clearly illustrates the
        anomaly detection function's execution.}
\end{frame}

\begin{frame}<1-2>[label=algorithm-animation]{Animation}
        {A short animation illustrating the algorithm execution}
    \centering
    \movie[once,showcontrols]
        {\includegraphics[width=0.8\textwidth]{pca/testoutrank}}
        {animation/testoutrank.avi}

    \note<1>{This slide shows a short animation, illustrating the execution of
        the chosen anomaly detection method on a data set of 441 2-dimensional
        vectors.

        In this animation, the yellow points indicate the current block that is
        being processed. Each vector within the block is compared to all other
        vectors in the data set, illustrated by the blue marker in the
        animation.

        Each vector in the current block is given a score, a monotonic
        decreasing function that provides a heuristic to estimate if any given
        vector may be an outlier.}

    \note<2>{The current `best' outliers are highlighted in green in the
        animation. As the algorithm processes further blocks of data, the set of
        current outliers converges to the final set of outliers.

        If the score for any vector in the current block becomes less than the
        lowest score from the set of current outliers, then this vector is no
        longer considered to be a possible outlier, and is pruned from the data
        set. The pruned vectors appear in red in the animation.}
\end{frame}
