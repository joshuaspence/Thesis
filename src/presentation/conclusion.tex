%%%%%%%%%%%%%%%%%%%%%%%%%%%%%%%%%%%%%%%%%%%%%%%%%%%%%%%%%%%%%%%%%%%%%%%%%%%%%%%%
% Conclusion
%%%%%%%%%%%%%%%%%%%%%%%%%%%%%%%%%%%%%%%%%%%%%%%%%%%%%%%%%%%%%%%%%%%%%%%%%%%%%%%%
% In the conclusion, it is important to address the aims of the Thesis.
\begin{frame}[label=conclusion]{Conclusion}
    % Answers aims
    % Use quantitative numbers
    % Provide a verdict
    \begin{enumerate}
        \item<1-> Investigated the performance benefits that can be obtained
            from reconfigurable computing.
        \note<1>{In this \thesis{}, I was able to investigate the performance
            benefits that can be obtained by applying reconfigurable computing
            design principles to an anomaly detection algorithm using commute
            time and eigenspace embedding.}

        \bigskip
        \item<2-> {Whilst results were not conclusive, there is still further
            work to be completed in this area.}
        \note<2>{Whilst results were not conclusive, there is still further
            work to be completed in this area.

            \begin{itemize}
                \item Optimizations to reduce hardware clock period and latency.
                \item Additional efforts to reduce the number of required data
                    transfers.
                \item Further experimentation with the discussed design to
                    accurately measure the performance gains from
                    parallelisation.
            \end{itemize}}

        \bigskip
        \item<3-> The hardware framework designed has potential applications in
            a wide range of computational problems involved pairwise
            computations on large and multidimensional data sets.
        \note<3>{The hardware framework designed has potential applications in a
            wide range of computational problems involved pairwise computations
            on large and multidimensional data sets.}
        \end{enumerate}
\end{frame}

%%%%%%%%%%%%%%%%%%%%%%%%%%%%%%%%%%%%%%%%%%%%%%%%%%%%%%%%%%%%%%%%%%%%%%%%%%%%%%%%
% Questions
%%%%%%%%%%%%%%%%%%%%%%%%%%%%%%%%%%%%%%%%%%%%%%%%%%%%%%%%%%%%%%%%%%%%%%%%%%%%%%%%
\begin{frame}[label=questions]{Questions}\relax
    {\Huge Questions?}
\end{frame}
