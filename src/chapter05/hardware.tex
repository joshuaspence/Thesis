%%%%%%%%%%%%%%%%%%%%%%%%%%%%%%%%%%%%%%%%%%%%%%%%%%%%%%%%%%%%%%%%%%%%%%%%%%%%%%%%
% Introduction
%%%%%%%%%%%%%%%%%%%%%%%%%%%%%%%%%%%%%%%%%%%%%%%%%%%%%%%%%%%%%%%%%%%%%%%%%%%%%%%%
\subsection{Introduction}
\label{hardware:introduction}
Throughout this project, I have been using a \Xilinx{} Zynq-7000 EPP, which is
an ``all programmable \gls{SoC}''. This device contains a complete \ARM{}
processing system, including a dual \ARM{} \Cortex{} A9 processor, integrated
memory controllers and peripherals and tightly integrated \gls{FPGA}
programmable logic.

The processor supports single and double processor floating point arithmetic.

High bandwidth memory, with 32KB of L1 cache per core plus a unified 512KB L2
cache.

8 \gls{DMA} channels.

Open Standard Interconnect Enabled by AXI
* High Bandwidth Interconnect between Processing System and Programmable Logic
* ACP port for enhanced Hardware Acceleration and cache coherency for additional Soft processors

Programmable logic
* 28K - 235K logic cells
* 430K - 3.5M equivalent \gls{ASIC} gates
* Over 3000 internal interconnects, supporting up to 100Gb of bandwidth
